% @[10]@| Copyright © 1992–2023 by Steven L. Willoughby (AKA MadScienceZone)
% @[11]@| steve@madscience.zone (previously AKA Software Alchemy),
% @[12]@| Aloha, Oregon, USA. All Rights Reserved.
% @[13]@| Distributed under the terms and conditions of the BSD-3-Clause
% @[14]@| License as described in the accompanying LICENSE file distributed
% @[15]@| with GMA.
% @[16]@|
% @[20]@| Redistribution and use in source and binary forms, with or without
% @[21]@| modification, are permitted provided that the following conditions
% @[22]@| are met:
% @[23]@| 1. Redistributions of source code must retain the above copyright
% @[24]@|    notice, this list of conditions and the following disclaimer.
% @[25]@| 2. Redistributions in binary form must reproduce the above copy-
% @[26]@|    right notice, this list of conditions and the following dis-
% @[27]@|    claimer in the documentation and/or other materials provided
% @[28]@|    with the distribution.
% @[29]@| 3. Neither the name of the copyright holder nor the names of its
% @[30]@|    contributors may be used to endorse or promote products derived
% @[31]@|    from this software without specific prior written permission.
% @[32]@|
% @[33]@| THIS SOFTWARE IS PROVIDED BY THE COPYRIGHT HOLDERS AND
% @[34]@| CONTRIBUTORS “AS IS” AND ANY EXPRESS OR IMPLIED WARRANTIES,
% @[35]@| INCLUDING, BUT NOT LIMITED TO, THE IMPLIED WARRANTIES OF
% @[36]@| MERCHANTABILITY AND FITNESS FOR A PARTICULAR PURPOSE ARE
% @[37]@| DISCLAIMED. IN NO EVENT SHALL THE COPYRIGHT HOLDER OR CONTRIBUTORS
% @[38]@| BE LIABLE FOR ANY DIRECT, INDIRECT, INCIDENTAL, SPECIAL, EXEMPLARY,
% @[39]@| OR CONSEQUENTIAL DAMAGES (INCLUDING, BUT NOT LIMITED TO,
% @[40]@| PROCUREMENT OF SUBSTITUTE GOODS OR SERVICES; LOSS OF USE, DATA, OR
% @[41]@| PROFITS; OR BUSINESS INTERRUPTION) HOWEVER CAUSED AND ON ANY
% @[42]@| THEORY OF LIABILITY, WHETHER IN CONTRACT, STRICT LIABILITY, OR
% @[43]@| TORT (INCLUDING NEGLIGENCE OR OTHERWISE) ARISING IN ANY WAY OUT OF
% @[44]@| THE USE OF THIS SOFTWARE, EVEN IF ADVISED OF THE POSSIBILITY OF
% @[45]@| SUCH DAMAGE.
% @[46]@|
% @[50]@| This software is not intended for any use or application in which
% @[51]@| the safety of lives or property would be at risk due to failure or
% @[52]@| defect of the software.
%
% Common manual typesetting macros
% Steve Willoughby <steve@madscience.zone> 12/2009, updated constantly through 2017
%
% Provides the following (see below for more details)
% COMMANDS:
%  \acronym[*]{TEXT}                        Typeset TEXT as an acronym (\mc)  + index automatically unless *
%  \Acronym[*]{CAPITAL}{TEXT}               Like acronym but typeset first arg in regular type (for capitalization)
%  \ix{text}                                Typeset <text> plus do \index{<text>}
%  \LJimg[opts]{filename}                   Include image (wraps \epsfig)  opts: width= etc
%  \LJquote[*!][subtext]{[/@]author}{text}  Epigraph (*=wide, !=narrow)
%  \LJquotex[subtext]{[/@]author}{text}{w}  Eipgraph (random paragraph width)
%                                             {/author}  suppress indexing
%                                             {@author}  suppress "Last, First"
%                                             {x!y}      typeset x, index as y
%                                             {x!}       typeset x, do not index
%  \LJqindex*                               Suppress author indexing for next \LJquote
%  \LJqindex{entry}                         Index next \LJquote as <entry>, typeset author as-is in \LJquote arg
%  \LJversal{versal}{word+leader}           "Initial" paragraph after \LJstart
%  \mc{TEXT}                                Set TEXT in medium caps
%  \Qauthor[subtext]{author}                Give credit to author of quotation.
%  \starsep[char]                           Decorative * * * * separator
%  \strong{text}                            Typeset with strong emphasis
%  \TM                                      Trademark symbol
%  \URL{url}                                Set url name (tt font)
%  \Var[*]{name}                            Variable (*=bracketed)
%  \work[*]{title}                          Cited work name (Italics)  + index automatically unless *
%
% ENVIRONMENTS:
%  SourceCode                               Print block of source code in a box
%  Coding                                   Use |verbatim| and \Var in env
%                                            also \Var* puts <> around variable name
\documentclass[letterpaper,twoside,onecolumn,openright,final          ]{memoir}
%    vs.       others...   oneside twocolumn openleft  draft showtrims
%                          twoside onecolumn openany   ms
%                                            openright final 
%\usepackage{etex}
%\reserveinserts{28}
\usepackage{fontawesome}
\usepackage{ifpdf}
\usepackage{multirow}
\usepackage{ifthen}
\usepackage{relsize}
\usepackage{environ}
%\usepackage{bigfoot}
\usepackage[normalem]{ulem}
\usepackage{tcolorbox}
\usepackage{suffix}
\usepackage{wrapfig}
\usepackage{menukeys}
%\usepackage{epsfig}
\usepackage{textcomp}
\usepackage{lettrine}
\usepackage{bytefield}
\usepackage{amsmath}
\usepackage{amssymb}
\usepackage{bbding}
\usepackage{substr}
\usepackage[T1]{fontenc}
\usepackage{tgschola}
\usepackage{ucs}
\usepackage{siunitx}
\usepackage{colortbl}
\usepackage{longtable}
\usepackage{color}
\usepackage{calc}
\usepackage{graphicx}
\usepackage{soul}
\usepackage{xfrac}
\usepackage{wallpaper}
\usepackage{tikz}
\usepackage{verbatim}
\usepackage{algpseudocode}
\usepackage{algorithm}
\usepackage[colorlinks=true,linkcolor=blue]{hyperref}
\usepackage{enumitem}
\usepackage{wasysym}
\makeindex
\newcommand\ix[1]{#1\index{#1}}
%\makeglossary
%\showindexmarks
\nouppercaseheads
%\hideindexmarks
%
% Page Layout Parameters
%
\setstocksize{11in}{8.5in}        % printing on letter stock  8.5x11
%\settrimmedsize{234mm}{180mm}{*}  % trimmed paper dimensions ~7.5x9.5
%
% old sizes
%
%\settrimmedsize{234mm}{184mm}{*}  % trimmed paper dimensions ~7.5x9.5
%\settrims{23mm}{35mm}             % offset into stock of trimmed paper
%\settypeblocksize{182mm}{32pc}{*} % dimensions of type block
%\setlrmargins{*}{27mm}{*}         % margins 27mm from outer edge
%\setulmargins{*}{27mm}{*}         % margins 27mm from bottom edge
%\setheaderspaces{*}{12mm}{*}      % 12mm between header and text baselines
%\setmarginnotes{5mm}{15mm}{2mm}   % margin notes 15mm wide, 5mm from text, 2mm from each other
%
% new sizes
%
%%\settrimmedsize{257mm}{178mm}{*}  % trimmed paper dimensions 
%\settrimmedsize{257mm}{180.5mm}{*}% trimmed paper dimensions 
%%\settrims{12mm}{40mm}             % offset into stock of trimmed paper
%\settrims{12mm}{37mm}             % offset into stock of trimmed paper
%\settypeblocksize{205mm}{31pc}{*} % dimensions of type block
%\setlrmargins{*}{27mm}{*}         % margins 27mm from outer edge
%\setulmargins{*}{27mm}{*}         % margins 27mm from bottom edge
%\setheaderspaces{*}{12mm}{*}      % 12mm between header and text baselines
%\setmarginnotes{5mm}{15mm}{2mm}   % margin notes 15mm wide, 5mm from text, 2mm from each other
\fixpdflayout
%
\trimFrame
\checkandfixthelayout

  \definecolor{chapterboxcolor}{rgb}{.647,.129,.149}%
%\newcommand\defaultchaptercolor{%
%  \definecolor{chapterboxcolor}{rgb}{.051,.278,.631}%
%}
%\defaultchaptercolor
\makeatletter
\newlength\dlf@normtxtw
\setlength\dlf@normtxtw{\textwidth}
\def\myhelvetfont{\def\fdefault{mdput}}
\newsavebox{\feline@chapter}
\newcommand\feline@chapter@marker[1][4cm]{%
\sbox\feline@chapter{%
\resizebox{!}{#1}{\fboxsep=1pt%
%\colorbox{chapterboxcolor}{\color{white}\bfseries\sffamily\thechapter}%
\colorbox{chapterboxcolor}{\color{white}\bfseries\thechapter}%
}}%
\rotatebox{90}{%
\resizebox{%
\heightof{\usebox{\feline@chapter}}+\depthof{\usebox{\feline@chapter}}}%
{!}{\scshape\so\@chapapp}}\quad%
\raisebox{\depthof{\usebox{\feline@chapter}}}{\usebox{\feline@chapter}}%
}
\newcommand\feline@chm[1][4cm]{%
\sbox\feline@chapter{\feline@chapter@marker[#1]}%
\makebox[0pt][l]{% aka \rlap
\makebox[1cm][r]{\usebox\feline@chapter}%
}}
\makechapterstyle{daleif1}{
\renewcommand\chapnamefont{\normalfont\Large\scshape\raggedleft\so}
\renewcommand\chaptitlefont{\normalfont\huge\bfseries\scshape\color{chapterboxcolor}}
\renewcommand\chapternamenum{}
\renewcommand\printchaptername{}
\renewcommand\printchapternum{\null\hfill\feline@chm[2.5cm]\par}
\renewcommand\afterchapternum{\par\vskip\midchapskip}
\renewcommand\printchaptertitle[1]{\chaptitlefont\raggedleft ##1\par}
}
\makeatother
\chapterstyle{daleif1}


%\chapterstyle{veelo}
%\stockletter      % 8.5 x 11  in stock paper size
%\pagestatement    % 8.5 x 5.5 in page size
%\settrimmedsize{9in}{6in}{*}
%\settrims{1in}{1.25in}
%\setlrmarginsandblock{.5in}{.5in}{*}
%\setulmarginsandblock{1in}{1in}{*}
%\trimFrame
%\quarkmarks
%
% Footnote Setup
%
%
% Fix table of contents spacing
% @pnumwidth [1.55em] width of box holding page numbers in TOC
% @tocrmarg  [2.55em] title text indentation from right margin
% @dotsep    [4.5]    math units of space between TOC leader dots.
%
\makeatletter
\renewcommand{\@pnumwidth}{2.5em}
\makeatother
%
%
\begin{document}
\setlength{\epigraphrule}{0pt}
\newcommand\Qauthor[2][]{%
  \\%
  \textnormal{---#2%
    \ifthenelse{\equal{#1}{}}{}{%
      \\%
      {\footnotesize #1}%
    }%
  }%
}
%
% Indexing
% \ix{<text>} typesets <text> and indexes it too (identical in both
%             cases--to get something different, use \index explicitly).
%
% \LJquote[subtext]{author}{quote text}
%    Epigraph (a single quote).  The optional subtext is set smaller
%    in upright shape under the author.  Authors are given an em-dash
%    in caps and small caps fully flush right.  The quote text itself
%    is in Italics in a 40% width paragraph which is itself against the
%    right margin.
%
%    The author's name is indexed.  The following special syntax is
%    recognized for the {author} parameter:
%XXX    <text>      - index <text> after trying to change* to Lastname, First
%XXX    @<text>     - index <text> as-is*
%XXX    /<text>     - typeset <text> but do NOT index (use \index manually)
%
%XXX    <text>!<i>  - typeset as <text>, index as <i> using all the usual
%                     indexing controls in <i>.
%XXX    <text>!     - typeset <text> but do NOT index.
%
%    (*These functions are handled by external scripts which prepare the
%    indexes)
%
% \LJquote*[subtext]{author}{quote text}
%    Alternative for wider quotations (uses 60% width).
%
% \LJquote![subtext]{author}{quote text}
%    Alternative for narrow quotations (uses 25% width).
%
% \LJquotex[subtext]{author}{quote text}{width}
%    Extended and generalized form allowing you to specify your own
%    paragraph width.
%
% \index{...}	Index 
%				entry
%				entry!subentry
%				entry!subentry!subsubentry
%				entry@formatted-entry
%				entry|command			page# in \command{...}
%				entry|see {other-entry}
%				entry|seealso{other-entry}
%				entry@form!sub@form
%				entry|(   ... entry|)    	mark page range
%				"! "@ "| ""			literal !, @, |, "
%
           \newcommand\LJquote[3][]{\LJquotex[#1]{#2}{#3}{0.400\textwidth}}
\WithSuffix\newcommand\LJquote*[3][]{\LJquotex[#1]{#2}{#3}{0.60\textwidth}}
\WithSuffix\newcommand\LJquote![3][]{\LJquotex[#1]{#2}{#3}{0.25\textwidth}}
%
\newcommand\SuppressLjQuoteIndexing{norm}
\newcommand\LJqindex[1]{%
%  \index[quotes]{#1}%
  \index{$1}%
  \renewcommand\SuppressLjQuoteIndexing{nil}%
}
\WithSuffix\newcommand\LJqindex*{\renewcommand\SuppressLjQuoteIndexing{nil}}

\newcommand\LJquotex[4][]{%
  \setlength{\epigraphwidth}{#4}%
  \ifthenelse{\equal{\SuppressLjQuoteIndexing}{norm}}{%
    %\index[quotes]{#2}%
    \index{#2}%
    \IfCharInString{@}{#2}{%
      \IfBeforeSubStringEmpty{@}{#2}{%
%        \def\LJqAuthor{\BehindSubString{@}{#2}}%
	 \def\LJqAuthor{#2}%
      }{%
        \def\LJqAuthor{#2}%
      }%
    }{%
      \def\LJqAuthor{#2}%
    }%
  }{%
    \renewcommand\SuppressLjQuoteIndexing{norm}%
    \def\LJqAuthor{#2}%
  }%
  \ifthenelse{\equal{\LJqAuthor}{}}{%
    \epigraph{{\em #3}}{}}{
    \ifthenelse{\equal{#1}{}}{%
      \epigraph{{\em #3}}{{---\LJqAuthor}}%
    }{%
      \epigraph{{\em #3}}{{---\LJqAuthor}\\
{\footnotesize #1}}%
    }%
  }%
}

% \LJqindex*    Do not index author name at all for next quote
% \LJqindex{@i} Use <i> as index entry for next \LJquote
% \LJqindex{i}  Use <i> as index entry for next \LJquote, changing to Last, First format
% \LJquote[work]{author}{text}  Set quote and index author in Last, First format
% \LJquote[work]{@author}{text} Set quote and index author as-is

\maxsecnumdepth{chapter}
% SourceCode environment:
%   Typesets source code in a verbatim box.
%
\newenvironment{SourceCode}{\boxedverbatim}{\endboxedverbatim}
\newenvironment{SourceCode*}{\begin{tcolorbox}\verbatim}{\endverbatim\end{tcolorbox}}
%
% Coding environment:
%   Code verbatim inside |bars|
%   Variables in \Var[*]{...} command  * form uses <> brackets around name
%
% Coding* puts the text in a colored box
% Coding! is like Coding* but uses ` instead of | to toggle verbatim text
% Coding: is like Coding! but is flush left
%   
\newenvironment{Coding}{\MakeShortVerb{\|}}{\DeleteShortVerb{\|}}
\newenvironment{Coding*}{\begin{tcolorbox}\MakeShortVerb{\|}}{\DeleteShortVerb{\|}\end{tcolorbox}}
\newenvironment{Coding!}{\begin{tcolorbox}\MakeShortVerb{\`}}{\DeleteShortVerb{\`}\end{tcolorbox}}
\newenvironment{Coding:}{\begin{tcolorbox}\MakeShortVerb{\`}\begin{flushleft}}{\end{flushleft}\DeleteShortVerb{\`}\end{tcolorbox}}
\newcommand\Var[1]{{\itshape #1\/}}
\WithSuffix\newcommand\Var*[1]{$\langle${\itshape #1\/}$\rangle$}
%
% \LJversal{versal}{first word(s)}
%   Begin a chapter paragraph.  This is just like \LJstart but without the
%   meta-post information being printed.
%   (In fact, \LJstart calls \LJversal to do the initial text.)
%
\newcommand\LJversal[2]{%
 \ifthenelse{\equal{#1}{A}}{%
  \lettrine[lines=3,findent=-1.1em,nindent=.9em,slope=.6em]{#1}{#2}%---------------A
 }{%
  \ifthenelse{\equal{#1}{B}}{%
   \lettrine[lines=3,findent=-.4em]{#1}{#2}%---------------------------------------B
  }{%
   \ifthenelse{\equal{#1}{C}}{%
    \lettrine[lines=3,findent=-.2em]{#1}{#2}%--------------------------------------C
   }{%
    \ifthenelse{\equal{#1}{D}}{%
     \lettrine[lines=3,findent=-.4em]{#1}{#2}%-------------------------------------D
    }{%
     \ifthenelse{\equal{#1}{E}}{%
      \lettrine[lines=3,findent=-.2em]{#1}{#2}%------------------------------------E
     }{%
      \ifthenelse{\equal{#1}{G}}{%
       \lettrine[lines=3,findent=-.3em]{#1}{#2}%-----------------------------------G
      }{%
       \ifthenelse{\equal{#1}{I}}{%
        \lettrine[lines=3,findent=-.1em]{#1}{#2}%----------------------------------I
       }{%
        \ifthenelse{\equal{#1}{K}}{%
         \lettrine[lines=3,findent=+.1em]{#1}{#2}%---------------------------------K
        }{%
         \ifthenelse{\equal{#1}{L}}{%
          \lettrine[lines=3,findent=-1.1em,nindent=1.5em]{#1}{#2}%-----------------L
         }{%
          \ifthenelse{\equal{#1}{O}}{%
           \lettrine[lines=3,findent=-.3em]{#1}{#2}%-------------------------------O
          }{%
           \ifthenelse{\equal{#1}{Q}}{%
            \lettrine[lines=3,findent=-.3em]{#1}{#2}%------------------------------Q
           }{%
            \ifthenelse{\equal{#1}{R}}{%
             \lettrine[lines=3,findent=-.3em]{#1}{#2}%-----------------------------R
            }{%
             \ifthenelse{\equal{#1}{S}}{%
              \lettrine[lines=3,findent=-.2em]{#1}{#2}%----------------------------S
             }{%
              \ifthenelse{\equal{#1}{V}}{%
               \lettrine[lines=3,findent=+.1em,nindent=0em,slope=-.5em]{#1}{#2}%---V
              }{%
               \ifthenelse{\equal{#1}{W}}{%
                \lettrine[lines=3,nindent=-.1em,slope=-.4em]{#1}{#2}%--------------W
               }{%
                \ifthenelse{\equal{#1}{Y}}{%
                 \lettrine[lines=3,findent=+.1em,nindent=0em,slope=-.5em]{#1}{#2}%-Y
                }{%
                 \lettrine[lines=3]{#1}{#2}%---------------------------------------*
	        }%
               }%
	      }%
	     }%
            }%
           }%
          }%
         }%
        }%
       }%
      }%
     }%
    }%
   }%
  }%
 }%
}

%
% \LJimg[options]{filename}
%   Place an inline EPS image here in the text (this needs refinement)
%
\newcommand\LJimg[2][]{%
  \ifthenelse{\equal{#2}{}}{%
    \emph{XXX Image? #1 #2}%
    \typeout{Missing image name in LJimg!}%
  }{%
    \ifthenelse{\boolean{pdf}}{%
      \ifthenelse{\equal{#1}{}}{%
        %\epsfig{file=images/#2, width=\textwidth}%
	\includegraphics[width=\textwidth]{images/#2}%
      }{%
        %\epsfig{file=images/#2, #1}%
	\includegraphics[#1]{images/#2}%
      }%
    }{%
      \ifthenelse{\equal{#1}{}}{%
        %\epsfig{file=eps/#2.eps, width=\textwidth}%
	\includegraphics[width=\textwidth]{eps/#2.eps}%
      }{%
        %\epsfig{file=eps/#2.eps, #1}%
	\includegraphics[#1]{eps/#2.eps}%
      }%
    }%
  }%
}% 
%
%
% \URLref[*][text]{url}
%   Refer to a URL somehow (as <a href="url">text</a> would have been)
%   Starred form uses 2nd level of footnotes (for use within a footnote)
%
\newcommand\URLref[2][]{#1\footnote{\URL{#2}}}
\WithSuffix\newcommand\URLref*[2][]{#1\footnoteB{\URL{#2}}}
%
% \acronym{letters}
% \mc{letters}
%   Typesets all-caps in slightly smaller font.  \mc just does the
%   typesetting, while \acronym also adds an entry in the main index
%   for the term (maybe someday we'll add to the glossary as well).
%
\newcommand\acronym[1]{{\smaller #1}\index{#1@\acronym*{#1}}}
\newcommand\Acronym[2]{#1{\smaller #2}\index{#1#2@#1{\smaller#2}}}
\WithSuffix\newcommand\acronym*[1]{{\smaller #1}}
\WithSuffix\newcommand\Acronym*[2]{#1{\smaller #2}}
\newcommand\mc[1]{{\smaller #1}}
%
% \starsep[symbol]
%   Star paragraph section separator.  The default symbol is
%   a fancy asterisk from the dingbats font.
%
\newcommand\starsep[1][\Asterisk]{
	\medskip
	\centerline{#1\qquad#1\qquad#1\qquad#1}
	\medskip
}

%
% unfinished
%
%\tracingmacros=1
\newcommand\strong[1]{{\bfseries #1}}
\newcommand\TM{\texttrademark}
\newcommand\work[1]{\emph{#1}\index{#1@\work*{#1}}}
\WithSuffix\newcommand\work*[1]{\emph{#1}}
\WithSuffix\newcommand\work:[2]{\work*{#1}\index{#2@\work*{#2}}}
%\WithSuffix\newcommand\worki[2]{\emph{#1 #2}\index{#2, #1@\emph{#2, #1}}}
\newcommand\URL[1]{{\ttfamily #1}}

\newenvironment{LJwrapimg}[3]{%
 \begingroup\setlength\intextsep{0pt}%
  \begin{wrapfigure}{#1}{#2}%
   \LJimg[width=#2]{#3}%
  \end{wrapfigure}%
}{%
 \par\endgroup%
}

\newcommand\z[1]{{\ttfamily#1}}
\newcommand\zb[1]{{\ttfamily\bfseries#1}}

% Based on post from tex.stackexchange.com from Gonzalo Medina
%\newlength\widest
%\makeatletter
%\NewEnviron{ldescription}{%
% \vbox{%
%  \global\setlength\widest{0pt}%
%  \def\item[##1]{%
%   \settowidth\@tempdima{\textbf{##1}}%
%   \ifdim\@tempdima>\widest\global\setlength\widest{\@tempdima}\fi%
%  }%
%  \setbox0=\hbox{\BODY}%
% }
% \begin{description}[
%   leftmargin=\dimexpr\widest+0.5em\relax,
%   labelindent=0pt,
%   labelwidth=\widest]
%  \BODY
%  \end{description}%
%}
%\makeatother
\newenvironment{ldescription}[1]{\begin{description}[leftmargin=#1+.5em,labelindent=0pt,labelwidth=#1]}{\end{description}}
%
% the NewContent environment shows material recently added to the document.
%
\newenvironment{NewContent}{\begingroup\color{blue}}{\endgroup}
