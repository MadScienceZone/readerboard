\input common
\frontmatter
\definecolor{reservedslot}{gray}{0.8}
\newcommand\api{\acronym{API}}
\newcommand\pc{\acronym{PC}}
\newcommand\cli{\acronym{CLI}}
\newcommand\ascii{\acronym{ASCII}}
\newcommand\led{\acronym{LED}}
\newcommand\codetype[1]{\z{#1}}
\newcommand\ixz[1]{\index{#1@\z{#1}}\z{#1}}
\newcommand\tUnused{\cellcolor{gray!50}}
\newcommand\tControl{\cellcolor{yellow!50}}
\newcommand\tForbidden{\cellcolor{red!50}}
\newcommand\tSpecial{\cellcolor{blue!25}}
%\colorlet{tableheader}{blue!40}
%\colorlet{tablesubhead}{blue!20}
%\colorlet{recordbox}{blue!15}
%\colorlet{recordtext}{black}
%\usetikzlibrary{positioning,shapes,shadows,arrows}
%\tikzstyle{program}=[rectangle, draw=black, rounded corners, fill=recordbox, drop shadow,
%	anchor=north, text=recordtext, text width=2cm]
%\tikzstyle{instance}=[rectangle, draw=black, rounded corners, fill=green!15, drop shadow,
%	anchor=north, text=black, text width=3cm]
%\tikzstyle{line}=[-, thick]
%\tikzstyle{myarrow}=[->, >=triangle 45, thick]
\hypersetup{
	pdftitle={Readerboard User's Guide},
	pdfkeywords={Open Source Readerboard Hardware and Software},
	pdfauthor={Steve Willoughby / Mad Science Zone},
	pdfsubject={Readerboard Project Documentation * (c) 2023 * Creative Commons Licensing (See Document for details)},
	colorlinks=true,
	linkcolor=blue!30!black,
}
\thispagestyle{empty}
	\begin{center}
		\Huge Readerboard \\ User's Guide \\
		WORKING\\DRAFT
	\end{center}

\vfill
%\end{flushright}
\newpage
The information in this document, and the hardware and software it describes, are hobbyist
works created as an educational exercise and as a matter of personal interest for recreational
purposes.

It is not to be considered an industrial-grade or retail-worthy product.
It is assumed that the user has the necessary understanding and skill to use it appropriately.  The author makes NO
representation as to suitability or fitness for any purpose whatsoever, and disclaims any and all liability or 
warranty to the full extent permitted by applicable law.  It is explicitly not designed for use where the safety
of persons, animals, property, or anything of real value depends on the correct operation of the software.

\strut\vfill

\begin{center}\bfseries
	For readerboard hardware version 1.0.0 and firmware version 0.0.0.
\end{center}

\strut\vfill

\noindent Copyright \copyright\ 2023 by Steven L. Willoughby
(aka MadScienceZone), Aloha, Oregon, USA. All Rights Reserved.
This document is released under the terms and conditions of the
Creative Commons ``Attribution-NoDerivs 3.0 Unported'' license.
In summary, you are free to use, reproduce, and redistribute this 
document provided you give full attribution to its author and do not
alter it or create derivative works from it.  See
\begin{center}
\href{http://creativecommons.org/licenses/by-nd-3.0}{http://creativecommons.org/licenses/by\-nd\-/\-3.0/} 
\end{center}
for the full set of licensing terms.

\begin{center}
\LJimg[width=.25in]{cc}\LJimg[width=.25in]{by}\LJimg[width=.25in]{nd}
\end{center}

\newpage
\tableofcontents
\newpage
\listoffigures
\listoftables
\mainmatter

%%%%%%%%%%%%%%%%%%%%%%%%%%%%%%%%%%%%%%%%%%%%%%%%%%%%%%%%%%%%%%%%%%%%%%%%%%%%%%%%%%%%%%%%%%%%%%%%%%%%
%  ____  ____   ___ _____ ___   ____ ___  _     
% |  _ \|  _ \ / _ \_   _/ _ \ / ___/ _ \| |    
% | |_) | |_) | | | || || | | | |  | | | | |    
% |  __/|  _ <| |_| || || |_| | |__| |_| | |___ 
% |_|   |_| \_\\___/ |_| \___/ \____\___/|_____|
%
\chapter{Protocol Description}\label{chap:protocol}
{\setlength{\epigraphwidth}{.5\textwidth}
\epigraph{I don't stand on protocol. Just call me your Excellency.}{---Henry Kissinger}}
\LJversal{T}{he control protocol} used to display information on the readerboard sign is very simple.
Commands are expressed largely in plain \ascii\ characters and are executed immediately
as they are received.\footnote{Technically, they may even be executed \emph{while} they
are being received.}
%It is not necessarily required for a command to be fully received first,
%so it is possible that the sign will have started operating on part of the command (e.g.,
%moving the current cursor column position) even if the rest of the command could not be
%performed.

In addition to plain, printable 7-bit \ascii\ characters, a few control codes are
recognized as described below. String data may include any 8-bit value except as otherwise
indicated.

\section{Command Addressing}
Commands may be received over the \acronym{USB} port (a point-to-point connection with a host computer)
or the RS-485 port (as part of a network of devices all connected to a single host computer's port).
Commands received over the \acronym{USB} port \mc{MAY} be prefixed by an address specifier as described
below. Those received over the RS-485 network \mc{MUST} have such a prefix.

The address specifier prefix has the form:
\begin{center}
\begin{bytefield}[endianness=little,bitwidth=0.11111\textwidth]{9}
	\bitbox[]{1}{\scriptsize0}&
	\bitbox[]{1}{\scriptsize1}&
	\bitbox[]{1}{\scriptsize\dots}&
	\bitbox[]{1}{\scriptsize$n$}&
	\bitbox[]{1}{\scriptsize$n+1$}\\
	\bitbox{1}{\z{/}} &
	\bitbox[tbl]{1}{\Var*{ad$_0$}}&
	\bitbox[tb]{1}{$\cdots$}&
	\bitbox[tbr]{1}{\Var*{ad$_{n-1}$}}&
	\bitbox{1}{\z{\$}}
\end{bytefield}
\end{center}

The \Var*{ad} parameters give the address(es) of the sign(s) which should obey the following
command. Each is a value from 0--63 encoded as shown in Table~\ref{tbl:int063}. If the list of
addresses is empty,
all signs should respond to the command. The address list
terminator may be a dollar-sign or escape
character (hex byte \z{1B}), indicated in the protocol description as \z{\$}.

If sent over the \acronym{USB} port, the address prefix is allowed but ignored, and the readerboard
acts as if all commands are addressed to it, since the commands are received over a private connection
between the host and that readerboard.

\section{Command Termination and Error Handling}
The command terminator is ``\z{\textasciicircum D}'' (hex byte value \z{04}). %We
%\emph{recommend} that every command---or group of commands which together represent
%a logical update to the sign---be followed by a \z{\textasciicircum D} character.
This character \emph{must} follow each readerboard command, since the interpretation
of the next command won't start until receipt of the 
\z{\textasciicircum D},
effectively
ignoring any data sent after the end of the command and before the terminating
\z{\textasciicircum D}.

The exception to this rule is the set of Busylight-compatible commands
\z*,
\z?,
\z{F},
\z{S},
and
\z{X}.
Since the Busylight unit did not use any command terminator in its protocol, the readerboard
unit won't demand one either, although we \emph{recommend} that you do anyway. At the very least,
you should have a
\z{\textasciicircum D}
character sent after a sequence of one or more Busylight-compatible commands.

%The individual commands are executed immediately upon receipt, without requiring the
%\z{\textasciicircum D} byte. For example, in the sequence ``\z{H1H2H3\textasciicircum D}''
%the first bar graph data point will be displayed as soon as ``\z{H1}'' is received.

In case of an error, such as the inability to parse the incoming command or an invalid
field value, the readerboard will signal the error condition by lighting the white and
both red discrete \led s, and will then ignore all input until the next
\z{\textasciicircum D} byte is received.

Whenever a \z{\textasciicircum D} byte is received, any command being parsed is aborted
and the sign's input parser is reset. Thus, this byte may be sent in case a partial command
has been sent but the host becomes aware that it cannot be completed. Note that the command
may have been partially acted upon by that point, so no assumption should be made as to the
sign's state.

%In the descriptions that follow, a trailing \z{\textasciicircum D} is shown at the end
%of each command to remind you of the recommendation to send this terminator, even though
%the commands themselves do not require them, strictly speaking.

\section{Command Summary}
The eight discrete \led s are intended for a simple display of status information
in a manner analogous to the Busylight project by the same author.\footnote{See
\href{https://github.com/MadScienceZone/busylight}{github.com/MadScienceZone/busylight}.}
To support this usage, the \z{F}, \z{S}, \z{X}, \z*, and \z? commands are recognized
in a manner compatible with how Busylight uses those same commands. These are categorized
as ``Busylight compatibility commands.'' Unlike all other commands listed here, these are
recognized regardless of case.  Since the readerboard has a power supply capable of illuminating
all of the status \led s at once,\footnote{The Busylight cannot, since it is powered
from the host computer's \acronym{USB} port.} a new command \z{L} is added which allows any
arbitrary pattern of steady \led s to be turned on.

The remaining commands are used for management of the matrix display. All commands are summarized in Table~\ref{tbl:commands}.
\begin{table}
	\begin{center}
		\begin{tabular}{cll}\toprule
			\multicolumn{1}{c}{\bfseries Command}&
			\multicolumn{1}{c}{\bfseries Description}&
			\multicolumn{1}{c}{\bfseries Notes}\\\midrule
			\z{*} & Strobe \led s in Sequence & [1]\\
			\z{?} & Query discrete \led\ status & [1] [2] [4]\\
			\z{F} & Flash \led s in Sequence & [1]\\
			\z{L} & Light one or more \led s steady & [3]\\
			\z{S} & Light one \led\ steady & [1]\\
			\z{X} & All \led s off & [1]\\
			\midrule
			\z{\textasciicircum D} & Abort/terminate command & \\
			\z{<} & Scroll text across display & \\
			\z{=} & Set operational parameters & [4]\\
			\z{@} & Move current column cursor & \\
			\z{A} & Select character font & \\
			\z{C} & Clear matrix display & \\
			\z{H} & Add histogram/bargraph data point & \\
			\z{I} & Draw bitmap graphic image & \\
			\z{Q} & Query matrix display statue & [2] [4]\\
			\z{T} & Display text on display & \\
			\bottomrule
			\multicolumn{3}{l}{\footnotesize [1] Busylight compatibile command}\\
			\multicolumn{3}{l}{\footnotesize [2] Sends response (\acronym{USB} only)}\\
			\multicolumn{3}{l}{\footnotesize [3] Busylight extension (not in original Busylight)}\\
			\multicolumn{3}{l}{\footnotesize [4] \acronym{USB} only}\\
		\end{tabular}
		\caption{Summary of All Commands\label{tbl:commands}}
	\end{center}
\end{table}

Although the rev 2 hardware supports the ability to enable the RS-485 transmitter and send data back onto the network,
this is not currently implemented by the firmware, and the intent is to have all devices listen passively to RS-485 traffic
at all times.

\section{\z{*}---Strobe Lights in Sequence}
\begin{center}
\begin{bytefield}[endianness=little,bitwidth=0.11111\textwidth]{8}
%	\bitheader{0-3} \\
	\bitbox[]{1}{\scriptsize0}&
	\bitbox[]{1}{\scriptsize1}&
	\bitbox[]{1}{\scriptsize2}&
	\bitbox[]{1}{\scriptsize3}&
	\bitbox[]{1}{\scriptsize\dots}&
	\bitbox[]{1}{\scriptsize$n$}&
	\bitbox[]{1}{\scriptsize$n+1$}&
	\bitbox[]{1}{\scriptsize$n+2$}\\
	\bitbox{1}{\z{*}} &
	\bitbox[tlb]{1}{\Var*{led$_0$}} &
	\bitbox[tb]{1}{\Var*{led$_1$}} &
	\bitbox[tb]{1}{\Var*{led$_2$}} &
	\bitbox[tb]{1}{$\cdots$} &
	\bitbox[tbr]{1}{\Var*{led$_{n-1}$}} &
	\bitbox{1}{\z\$}&
	\bitbox{1}{\z{\textasciicircum D}}
\end{bytefield}
\end{center}

Each \Var*{led} value is an \ascii\ digit character corresponding to a discrete
\led\ as shown in Table~\ref{tbl:lightcodes}. 

This command functions identically to the \z{F} command (see below), except that the lights
are ``strobed'' (flashed very briefly with a pause between each light in the sequence).

\section{\z<---Scroll Text Across Display}
\begin{center}
\begin{bytefield}[endianness=little,bitwidth=0.11111\textwidth]{9}
%	\bitheader{0-8} \\
	\bitbox[]{1}{\scriptsize0}&
	\bitbox[]{1}{\scriptsize1}&
	\bitbox[]{1}{\scriptsize2}&
	\bitbox[]{3}{\scriptsize\dots}&
	\bitbox[]{1}{\scriptsize$n$+1}&
	\bitbox[]{1}{\scriptsize$n$+2}&
	\bitbox[]{1}{\scriptsize$n$+3}\\
	\bitbox{1}{\z{<}} &
	\bitbox{1}{\Var*{loop}}&
	\bitbox{5}{\Var*{string}}&
	\bitbox{1}{\z{ESC}}&
	\bitbox{1}{\z{\textasciicircum D}}
\end{bytefield}
\end{center}

Displays the text \Var*{string} by scrolling it across the display from right to left.
If \Var*{loop} is ``\z.'', the text is only scrolled once; if it is ``\z{L}'' then it
repeatedly scrolls across the screen in an endless loop.

The text is rendered in the current font and may contain any 8-bit bytes except as otherwise
noted (but avoiding \ascii\ control codes is wise to be safe from conflict with
future control codes which may be added to the protocol). The string is terminated by an
escape character (hex byte \z{1B}), indicated in the protocol description as \z{ESC}.

The string may include control codes as listed in Table~\ref{tbl:controlcodes}.
\begin{table}
	\begin{center}
		\begin{tabular}{lll}\toprule
			\multicolumn{1}{c}{\bfseries Code} &
			\multicolumn{1}{c}{\bfseries Hex} &
			\multicolumn{1}{c}{\bfseries Description} \\\midrule
			\z{\textasciicircum C}\Var*{pos} & \z{03}\Var*{pos} & Move current column cursor to \Var*{pos}\\
			\z{\textasciicircum D} & \z{04} & Never allowed in strings (command terminator)\\
			\z{\textasciicircum F}\Var*{digit} & \z{06}\Var*{digit} & Switch current font\\
			\z{\textasciicircum H}\Var*{pos} & \z{08}\Var*{pos} & Move cursor left \Var*{pos} columns\\
			\z{\textasciicircum L}\Var*{pos} & \z{0C}\Var*{pos} & Move cursor right \Var*{pos} columns\\
			\z{\textasciicircum [} & \z{1B} & Never allowed in strings (string terminator)\\
			\bottomrule
		\end{tabular}
		\caption{Control Codes in String Values\label{tbl:controlcodes}}
	\end{center}
\end{table}

\section{\z{=}---Set Operational Parameters}
\begin{center}
\begin{bytefield}[endianness=little,bitwidth=0.11111\textwidth]{8}
%	\bitheader{0-3} \\
	\bitbox[]{1}{\scriptsize0}&
	\bitbox[]{1}{\scriptsize1}&
	\bitbox[]{1}{\scriptsize2}&
	\bitbox[]{1}{\scriptsize3}&
	\bitbox[]{1}{\scriptsize4}\\
	\bitbox{1}{\z{=}} &
	\bitbox{1}{\Var*{ad}}&
	\bitbox{1}{\Var*{uspd}}&
	\bitbox{1}{\Var*{rspd}}&
	\bitbox{1}{\z{\textasciicircum D}}
\end{bytefield}
\end{center}

This command sets a few operational parameters for the sign. Once set, these will be persistent across
power cycles and reboots.

If the \Var*{ad} parameter is ``\z{*}'' then the RS-485 interface is disabled entirely. Otherwise it is a
value from 0--63 encoded as described in Table~\ref{tbl:int063}. This enables the RS-485 interface and assigns
this sign's address to \Var*{ad}.

The baud rate for the \acronym{USB} and RS-485 interfaces is set by the \Var*{uspd} and \Var*{rspd} values
respectively. Each is encoded as per Table~\ref{tbl:baudcodes}.

This command may only be sent over the \acronym{USB} port.

By default, an unconfigured readerboard is set to 9,600 baud with the RS-485 port disabled.
\begin{table}
	\begin{center}
		\begin{tabular}{crl}\toprule
			\bfseries Code & \multicolumn{1}{c}{\bfseries Speed} \\\midrule
			0 & 300\\
			1 & 600\\
			2 & 1,200\\
			3 & 2,400\\
			4 & 4,800\\
			5 & 9,600 & (default)\\
			6 & 14,400\\
			7 & 19,200\\
			8 & 28,800\\
			9 & 31,250\\
			A & 38,400\\
			B & 57,600\\
			C & 115,200\\
		\bottomrule
		\end{tabular}
		\caption{Baud Rate Codes\label{tbl:baudcodes}}
	\end{center}
\end{table}

\section{\z?---Query Discrete \led\ Status}
\begin{center}
\begin{bytefield}[endianness=little,bitwidth=0.11111\textwidth]{2}
	\bitheader{0-1} \\
	\bitbox{1}{\z{?}} &
	\bitbox{1}{\z{\textasciicircum D}}
\end{bytefield}
\end{center}

This command causes the sign to send a status report back to the host to indicate
what the discrete \led s are currently showing. This response has the form:

\medskip

\begin{center}\begin{bytefield}[endianness=little,bitwidth=0.11111\textwidth]{9}
	\bitheader{0-8} \\
	\bitbox{1}{\z{L}} &
	\bitbox[tbl]{1}{\Var*{led$_0$}} &
	\bitbox[tb]{1}{\Var*{led$_1$}} &
	\bitbox[tb]{1}{\Var*{led$_2$}} &
	\bitbox[tb]{1}{\Var*{led$_3$}} &
	\bitbox[tb]{1}{\Var*{led$_4$}} &
	\bitbox[tb]{1}{\Var*{led$_5$}} &
	\bitbox[tb]{1}{\Var*{led$_6$}} &
	\bitbox[tbr]{1}{\Var*{led$_7$}} \\
	\bitbox{1}{\z{F}} &
	\bitbox{8}{flasher status (see below)} \\
	\bitbox{1}{\z{S}} &
	\bitbox{8}{strober status (see below)} \\
	\bitbox{1}{\z{\textbackslash n}}
\end{bytefield}
\end{center}

The flasher and strober status values are variable-width fields which indicate the
state of the flasher (see \z{F} command) and strober (see \z{*} command) functions.
In each case, if there is no defined sequence, the status field will be:

\medskip

\begin{center}\begin{bytefield}[endianness=little,bitwidth=0.11111\textwidth]{2}
	\bitheader{0-1} \\
	\bitbox{1}{\Var*{run}} &
	\bitbox{1}{\z{X}}
\end{bytefield}
\end{center}

\smallskip

\noindent Otherwise, the state of the flasher or strober unit is indicated by:

\medskip

\begin{center}\begin{bytefield}[endianness=little,bitwidth=0.11111\textwidth]{7}
%	\bitheader{0-6} \\
	\bitbox[]{1}{\scriptsize0}&
	\bitbox[]{1}{\scriptsize1}&
	\bitbox[]{1}{\scriptsize2}&
	\bitbox[]{1}{\scriptsize3}&
	\bitbox[]{1}{\scriptsize4}&
	\bitbox[]{1}{\scriptsize\dots}&
	\bitbox[]{1}{\scriptsize$n+3$}\\
	\bitbox{1}{\Var*{run}} &
	\bitbox{1}{\Var*{pos}} &
	\bitbox{1}{\z{@}} &
	\bitbox[tbl]{1}{\Var*{led$_0$}} &
	\bitbox[tb]{1}{\Var*{led$_1$}} &
	\bitbox[tb]{1}{$\cdots$} &
	\bitbox[tbr]{1}{\Var*{led$_{n-1}$}} 
\end{bytefield}
\end{center}

In either case, \Var*{run} is the \ascii\ character ``\z{0}'' if the unit is
stopped or ``\z{1}'' if it is currently running.  If there is a defined sequence,
\Var*{pos} indicates the 0-origin position within the sequence of the light currently
being flashed or strobed, encoded as described in Table~\ref{tbl:int063}. 
The \Var*{led} values are as given to the \z{F} or \z{*}
command that set the sequence.

Note that the \Var*{pos} value may be the character ``\z{X}'' to indicate the position
value of 40, so it is important to distinguish between the field value
``\Var*{run}\z{X}'' vs.\@ ``\Var*{run}\z{X@}\Var*{sequence}''.

The status message sent to the host is terminated by a newline character (hex byte \z{0A}),
indicated in the protocol description above as ``\z{\textbackslash n}''.
\begin{table}
	\begin{center}
		\begin{tabular}{cc|cc}\toprule
			\multicolumn{1}{c}{\bfseries Value} &
			\multicolumn{1}{c}{\bfseries Code} &
			\multicolumn{1}{c}{\bfseries Value} &
			\multicolumn{1}{c}{\bfseries Code} \\\midrule
			0--9 & \z0--\z9 & 17--42 & \z{A}--\z{Z} \\
			10 & \z: & 43 & \z[ \\
			11 & \z; & 44 & \z\textbackslash \\
			12 & \z< & 45 & \z] \\
			13 & \z= & 46 & \z\textasciicircum \\
			14 & \z> & 47 & \z{\_} \\
			15 & \z? & 48 & \z` \\
			16 & \z@ & 49--63 & \z{a}--\z{o} \\
			\bottomrule
		\end{tabular}

		{\footnotesize (Each code is the numeric value plus 48.)}
		\caption{\ascii\ Encoded Integer Values (0--63)\label{tbl:int063}}
	\end{center}
\end{table}

This command may only be sent on the \acronym{USB} port.

\section{\z{@}---Set Column Cursor Position}
\begin{center}
\begin{bytefield}[endianness=little,bitwidth=0.11111\textwidth]{3}
	\bitheader{0-2} \\
	\bitbox{1}{\z{@}} &
	\bitbox{1}{\Var*{pos}} &
	\bitbox{1}{\z{\textasciicircum D}}
\end{bytefield}
\end{center}

Sets the column cursor position to the value indicated by \Var*{pos}. See Table~\ref{tbl:int063}.

\section{\z{A}---Select Font}
\begin{center}
\begin{bytefield}[endianness=little,bitwidth=0.11111\textwidth]{3}
	\bitheader{0-2} \\
	\bitbox{1}{\z{A}} &
	\bitbox{1}{\Var*{digit}} &
	\bitbox{1}{\z{\textasciicircum D}}
\end{bytefield}
\end{center}

Sets the font to use for rendering text with the \z< and \z{T} commands.
The font codes for \Var*{digit} are listed in Table~\ref{tbl:fontcodes}.
The full text fonts support the printable \ascii\ characters plus a majority
of the Unicode glyphs with codepoints less than 256.  See Tables~\ref{tbl:font0}--\ref{tbl:font2}
for a complete font glyph listing with codepoint assignments.

\begin{table}
	\begin{center}
		\begin{tabular}{ll}\toprule
			\multicolumn{1}{c}{\bfseries Code} &
			\multicolumn{1}{c}{\bfseries Font Description} \\\midrule
			\z0 & Fixed-width 5$\times$7 matrix plus descenders in 8th row \\
			\z1 & Variable-width version of font 0 \\
			\z2 & Bold alphanumerics and special symbols \\
			\bottomrule
		\end{tabular}
		\caption{Font Codes\label{tbl:fontcodes}}
	\end{center}
\end{table}
\begin{table}
	\begin{center}
		\begin{tabular}{r|c|c|c|c|c|c|c|c|l}
			&\emph{0} &\emph{1} &\emph{2} &\emph{3}
			&\emph{4} &\emph{5} &\emph{6} &\emph{7}&\\\hline
			\emph{00x}&\tUnused&\tUnused&\tUnused&\tControl\tiny move&\tForbidden\tiny end&\tUnused&\tControl\tiny font&\tUnused&\multirow{2}{*}{\z{0}\emph{x}}\\\cline{1-9}
			\emph{01x}&\tControl\tiny back&\tUnused&\tUnused&\tUnused&\tControl\tiny forw&\tUnused&\tUnused&\tUnused&\\\hline
			\emph{02x}&\tUnused&\tUnused&\tUnused&\tUnused&\tUnused&\tUnused&\tUnused&\tUnused&\multirow{2}{*}{\z{1}\emph{x}}\\\cline{1-9}
			\emph{03x}&\tUnused&\tUnused&\tUnused&\tForbidden\tiny esc&\tUnused&\tUnused&\tUnused&\tUnused&\\\hline
			\emph{04x}&&!&\z"&\#&\$&\%&\&&\z'&\multirow{2}{*}{\z{2}\emph{x}}\\\cline{1-9}
			\emph{05x}&(&)&*&+&,&-&.&/&\\\hline
			\emph{06x}&0&1&2&3&4&5&6&7&\multirow{2}{*}{\z{3}\emph{x}}\\\cline{1-9}
			\emph{07x}&8&9&:&;&\z<&=&\z>&?&\\\hline
			\emph{10x}&@&A&B&C&D&E&F&G&\multirow{2}{*}{\z{4}\emph{x}}\\\cline{1-9}
			\emph{11x}&H&I&J&K&L&M&N&O&\\\hline
			\emph{12x}&P&Q&R&S&T&U&V&W&\multirow{2}{*}{\z{5}\emph{x}}\\\cline{1-9}
			\emph{13x}&X&Y&Z&[&\textbackslash&]&\textasciicircum&\_&\\\hline
			\emph{14x}&`&a&b&c&d&e&f&g&\multirow{2}{*}{\z{6}\emph{x}}\\\cline{1-9}
			\emph{15x}&h&i&j&k&l&m&n&o&\\\hline
			\emph{16x}&p&q&r&s&t&u&v&w&\multirow{2}{*}{\z{7}\emph{x}}\\\cline{1-9}
			\emph{17x}&x&y&z&\{&|&\}&\textasciitilde&///&\\\hline
			\emph{20x}&``&''&`&'&$\dagger$&$\ddagger$&\dots&$'$&\multirow{2}{*}{\z{8}\emph{x}}\\\cline{1-9}
			\emph{21x}&$''$&!!&$\overline{\hbox{\phantom{x}}}$&$\leftarrow$&$\rightarrow$&$\uparrow$&$\downarrow$&$\ne$&\\\hline
			\emph{22x}&$\le$&$\ge$&$\approx$&$\Gamma$&$\Delta$&$\Xi$&$\Pi$&$\Sigma$&\multirow{2}{*}{\z{9}\emph{x}}\\\cline{1-9}
			\emph{23x}&$\Omega$&$\pi$&$\rho$&$\sigma$&---&\textperthousand&\tUnused&\tUnused&\\\hline
			\emph{24x}&&!`&\textcent&\textsterling&\textcurrency&\textyen&\textbrokenbar&\S&\multirow{2}{*}{\z{A}\emph{x}}\\\cline{1-9}
			\emph{25x}&\tUnused&\textcopyright&\textordfeminine&\guillemotleft&$\neg$&-&\textregistered&\tUnused&\\\hline
			\emph{26x}&$^\circ$&$\pm$&$^2$&$^3$&\tUnused&$\mu$&\P&$\bullet$&\multirow{2}{*}{\z{B}\emph{x}}\\\cline{1-9}
			\emph{27x}&\tUnused&$^1$&\textordmasculine&\guillemotright&1/4&1/2&3/4&?`&\\\hline
			\emph{30x}&\`A&\'A&\^A&\~A&\"A&\AA&\AE&\c C&\multirow{2}{*}{\z{C}\emph{x}}\\\cline{1-9}
			\emph{31x}&\`E&\'E&\^E&\"E&\`I&\'I&\^I&\"I&\\\hline
			\emph{32x}&\DH&\~N&\`O&\'O&\^O&\~O&\"O&$\times$&\multirow{2}{*}{\z{D}\emph{x}}\\\cline{1-9}
			\emph{33x}&\O&\`U&\'U&\^U&\"U&\'Y&\TH&\ss&\\\hline
			\emph{34x}&\`a&\'a&\^a&\~a&\"a&\aa&\ae&\c c&\multirow{2}{*}{\z{E}\emph{x}}\\\cline{1-9}
			\emph{35x}&\`e&\'e&\^e&\"e&\`\i&\'\i&\^\i&\"\i&\\\hline
			\emph{36x}&\dh&\~n&\`o&\'o&\^o&\~o&\"o&$\div$&\multirow{2}{*}{\z{F}\emph{x}}\\\cline{1-9}
			\emph{37x}&\o&\`u&\'u&\^u&\"u&\'y&\th&\"y&\\\hline
%			&\emph{x}\z{0} &\emph{x}\z{1} &\emph{x}\z{2} &\emph{x}\z{3}
%			&\emph{x}\z{4} &\emph{x}\z{5} &\emph{x}\z{6} &\emph{x}\z{7}\\
			&\z{8} &\z{9} &\z{A} &\z{B}
			&\z{C} &\z{D} &\z{E} &\z{F}
		\end{tabular}
	\end{center}
	\caption{Font Table for Fonts \#0 and \#1\label{tbl:font0}}
\end{table}
\begin{table}
	\begin{center}
		\begin{tabular}{r|c|c|c|c|c|c|c|c|l}
			&\emph{0} &\emph{1} &\emph{2} &\emph{3}
			&\emph{4} &\emph{5} &\emph{6} &\emph{7}&\\\hline
			\emph{00x}&\tUnused&\tUnused&\tUnused&\tControl\tiny move&\tForbidden\tiny end&\tUnused&\tControl\tiny font&\tUnused&\multirow{2}{*}{\z{0}\emph{x}}\\\cline{1-9}
			\emph{01x}&\tControl\tiny back&\tUnused&\tUnused&\tUnused&\tControl\tiny forw&\tUnused&\tUnused&\tUnused&\\\hline
			\emph{02x}&\tUnused&\tUnused&\tUnused&\tUnused&\tUnused&\tUnused&\tUnused&\tUnused&\multirow{2}{*}{\z{1}\emph{x}}\\\cline{1-9}
			\emph{03x}&\tUnused&\tUnused&\tUnused&\tForbidden\tiny esc&\tUnused&\tUnused&\tUnused&\tUnused&\\\hline
			\emph{04x}&&\bfseries!&\tUnused&\tUnused&\tUnused&\tUnused&\tUnused&\tUnused&\multirow{2}{*}{\z{2}\emph{x}}\\\cline{1-9}
			\emph{05x}&\tUnused&\tUnused&\tUnused&\tUnused&,&\tUnused&.&\tUnused&\\\hline
			\emph{06x}&\bfseries 0&\bfseries 1&\bfseries 2&\bfseries 3&\bfseries 4&\bfseries 5&\bfseries 6&\bfseries 7&\multirow{2}{*}{\z{3}\emph{x}}\\\cline{1-9}
			\emph{07x}&\bfseries 8&\bfseries 9&\tUnused&\tUnused&\tUnused&\tUnused&\tUnused&\tUnused&\\\hline
			\emph{10x}&\textcopyright&\bfseries A&\bfseries B&\bfseries C&\bfseries D&\bfseries E&\bfseries F&\bfseries G&\multirow{2}{*}{\z{4}\emph{x}}\\\cline{1-9}
			\emph{11x}&\bfseries H&\bfseries I&\bfseries J&\bfseries K&\bfseries L&\bfseries M&\bfseries N&\bfseries O&\\\hline
			\emph{12x}&\bfseries P&\bfseries Q&\bfseries R&\bfseries S&\bfseries T&\bfseries U&\bfseries V&\bfseries W&\multirow{2}{*}{\z{5}\emph{x}}\\\cline{1-9}
			\emph{13x}&\bfseries X&\bfseries Y&\bfseries Z&\tUnused&\tUnused&\tUnused&\tUnused&\tUnused&\\\hline
			\emph{14x}&\tUnused&\faArrowLeft&\faArrowRight&\faArrowUp&\faArrowDown&$\nwarrow$&$\nearrow$&$\searrow$&\multirow{2}{*}{\z{6}\emph{x}}\\\cline{1-9}
			\emph{15x}&$\swarrow$&$\leftarrow$&$\rightarrow$&$\uparrow$&$\downarrow$&\raise1ex\hbox{\rule{1ex}{1ex}}\rule{1ex}{1ex}&\rule{1em}{1em}&\tUnused&\\\hline
			\emph{16x}&\checkmark&$\sqrt{}$&$\infty$&\OE&\oe&\texteuro&$\therefore$&$\because$&\multirow{2}{*}{\z{7}\emph{x}}\\\cline{1-9}
			\emph{17x}&\XSolidBold&$\blacktriangleleft$&$\blacktriangleright$&$\blacktriangle$&$\blacktriangledown$&$\updownarrow$&$\blacklozenge$&$\lozenge$&\\\hline
			\emph{20x}&$\lambda$&$\Theta$&$\Phi$&$\Psi$&\tUnused&\tUnused&\tUnused&\tUnused&\multirow{2}{*}{\z{8}\emph{x}}\\\cline{1-9}
			\emph{21x}&\tUnused&\tUnused&\tUnused&\tUnused&\tUnused&\tUnused&\tUnused&\tUnused&\\\hline
			\emph{22x}&\tUnused&\tUnused&\tUnused&\tUnused&\tUnused&\tUnused&\tUnused&\tUnused&\multirow{2}{*}{\z{9}\emph{x}}\\\cline{1-9}
			\emph{23x}&\tUnused&\tUnused&\tUnused&\tUnused&\tUnused&\tUnused&\tUnused&\tUnused&\\\hline
			\emph{24x}&&\tSpecial\tiny{TS1}&\tSpecial\tiny{TS2}&\tSpecial\tiny{TS3}&\tSpecial\tiny{TS4}&\tSpecial\tiny{TS5}&\tSpecial\tiny{TS6}&\tSpecial\tiny{TS7}&\multirow{2}{*}{\z{A}\emph{x}}\\\cline{1-9}
			\emph{25x}&\tSpecial\tiny{TS8}&\tUnused&\tUnused&\tUnused&\tUnused&\tUnused&\tUnused&\tUnused&\\\hline
%			\emph{26x}&\tUnused&\tUnused&\tUnused&\tUnused&\tUnused&\tUnused&\tUnused&\tUnused&\multirow{2}{*}{\z{B}\emph{x}}\\\cline{1-9}
%			\emph{27x}&\tUnused&\tUnused&\tUnused&\tUnused&\tUnused&\tUnused&\tUnused&\tUnused&\\\hline
%			\emph{30x}&\tUnused&\tUnused&\tUnused&\tUnused&\tUnused&\tUnused&\tUnused&\tUnused&\multirow{2}{*}{\z{C}\emph{x}}\\\cline{1-9}
%			\emph{31x}&\tUnused&\tUnused&\tUnused&\tUnused&\tUnused&\tUnused&\tUnused&\tUnused&\\\hline
%			\emph{32x}&\tUnused&\tUnused&\tUnused&\tUnused&\tUnused&\tUnused&\tUnused&\tUnused&\multirow{2}{*}{\z{D}\emph{x}}\\\cline{1-9}
%			\emph{33x}&\tUnused&\tUnused&\tUnused&\tUnused&\tUnused&\tUnused&\tUnused&\tUnused&\\\hline
%			\emph{34x}&\tUnused&\tUnused&\tUnused&\tUnused&\tUnused&\tUnused&\tUnused&\tUnused&\multirow{2}{*}{\z{E}\emph{x}}\\\cline{1-9}
%			\emph{35x}&\tUnused&\tUnused&\tUnused&\tUnused&\tUnused&\tUnused&\tUnused&\tUnused&\\\hline
%			\emph{36x}&\tUnused&\tUnused&\tUnused&\tUnused&\tUnused&\tUnused&\tUnused&\tUnused&\multirow{2}{*}{\z{F}\emph{x}}\\\cline{1-9}
%			\emph{37x}&\tUnused&\tUnused&\tUnused&\tUnused&\tUnused&\tUnused&\tUnused&\tUnused&\\\hline
%			&\emph{x}\z{0} &\emph{x}\z{1} &\emph{x}\z{2} &\emph{x}\z{3}
%			&\emph{x}\z{4} &\emph{x}\z{5} &\emph{x}\z{6} &\emph{x}\z{7}\\
			&\z{8} &\z{9} &\z{A} &\z{B}
			&\z{C} &\z{D} &\z{E} &\z{F}
		\end{tabular}

		\smallskip

		{\footnotesize TS\Var*{n} = Thin space of \Var*{n} pixels}
	\end{center}
	\caption{Font Table for Font \#2\label{tbl:font2}}
\end{table}
			

\section{\z{C}---Clear Matrix Display}
\begin{center}
\begin{bytefield}[endianness=little,bitwidth=0.11111\textwidth]{2}
	\bitheader{0-1} \\
	\bitbox{1}{\z{C}} &
	\bitbox{1}{\z{\textasciicircum D}}
\end{bytefield}
\end{center}

Clears the matrix display so that no \led s are illuminated. Does not affect
the discrete \led s.

\section{\z{F}---Flash Lights in Sequence}
\begin{center}
\begin{bytefield}[endianness=little,bitwidth=0.11111\textwidth]{8}
%	\bitheader{0-7} \\
	\bitbox[]{1}{\scriptsize0}&
	\bitbox[]{1}{\scriptsize1}&
	\bitbox[]{1}{\scriptsize2}&
	\bitbox[]{1}{\scriptsize3}&
	\bitbox[]{1}{\dots}&
	\bitbox[]{1}{\scriptsize$n$}&
	\bitbox[]{1}{\scriptsize$n+1$}&
	\bitbox[]{1}{\scriptsize$n+2$}\\
	\bitbox{1}{\z{F}} &
	\bitbox[tbl]{1}{\Var*{led$_0$}} &
	\bitbox[tb]{1}{\Var*{led$_1$}} &
	\bitbox[tb]{1}{\Var*{led$_2$}} &
	\bitbox[tb]{1}{$\cdots$} &
	\bitbox[tbr]{1}{\Var*{led$_{n-1}$}} &
	\bitbox{1}{\z\$}&
	\bitbox{1}{\z{\textasciicircum D}}
\end{bytefield}
\end{center}

Each \Var*{led} value is an \ascii\ digit character corresponding to a discrete
\led\ as shown in Table~\ref{tbl:lightcodes}. Note that the assignment of colors
to these \led s is dependent on your particular hardware being assembled that way.
As an open source project, of course, you (or whomever assembled the unit) may choose any
color scheme you like when building the board.
\begin{table}
	\begin{center}
		\begin{tabular}{cll}\toprule
			\multicolumn{1}{c}{\bfseries Code}&
			\multicolumn{1}{c}{\bfseries Light}&
			\multicolumn{1}{c}{\bfseries Color}\\\midrule
			\z0 & L$_0$ & white \\
			\z1 & L$_1$ & blue \\
			\z2 & L$_2$ & blue \\
			\z3 & L$_3$ & red \\
			\z4 & L$_4$ & red \\
			\z5 & L$_5$ & yellow \\
			\z6 & L$_6$ & yellow \\
			\z7 & L$_7$ & green \\
			\bottomrule
		\end{tabular}
		\caption{Discrete \led\ Codes and Colors\label{tbl:lightcodes}}
	\end{center}
\end{table}

Up to 64 \Var*{led} codes may be listed. The sign will cycle through the sequence, lighting each
specified \led\ briefly before moving on to the next one. The sequence is repeated
forever in a loop until an \z{L}, \z{S} or \z{X} command is received. 

If only one \Var*{led} is specified, that light will be flashed on and off.
Setting an empty
sequence (no codes at all) stops the flasher's operation.

The sequence is terminated by either a dollar-sign (``\z{\$}'') character or the
escape control character (hex byte \z{1B}), indicated in the protocol diagram above
simply as ``\z{\$}''.
			
This command may be given in upper- or lower-case (``\z{f}'' or ``\z{F}'').

\section{\z{H}---Draw Bar Graph Data Point}
\begin{center}
\begin{bytefield}[endianness=little,bitwidth=0.11111\textwidth]{3}
	\bitheader{0-2} \\
	\bitbox{1}{\z{H}} &
	\bitbox{1}{\Var*{n}} &
	\bitbox{1}{\z{\textasciicircum D}}
\end{bytefield}
\end{center}

This command is used to draw a bar-graph element. Repeating this command causes a
scrolling data display which shows a set of data samples over some period of time.
The value \Var*{n} is an \ascii\ digit character in the range ``\z0''--``\z8'', and
is drawn in the far-right column of the matrix display, as a column of \Var*{n} lights
stacked up from the bottom row (a value of 0 results in no lights, up to 8 which is a full
column of eight lights; a value of 9 is treated as if it were 8). All existing matrix data
are scrolled left one column.

\section{\z{I}---Draw Bitmap Image}
\begin{center}
\begin{bytefield}[endianness=little,bitwidth=0.11111\textwidth]{9}
	\bitheader{0-8} \\
	\bitbox{1}{\z{I}} &
	\bitbox{1}{\Var*{merge}} &
	\bitbox{1}{\Var*{pos}} &
	\bitbox{1}{\Var*{trans}} &
	\bitbox[tbl]{2}{\Var*{coldata$_0$}} &
	\bitbox[tb]{2}{\Var*{coldata$_1$}} &
	\bitbox[tb]{1}{$\cdots$} \\
	\bitbox[tbr]{2}{\Var*{coldata$_{n-1}$}} &
	\bitbox{1}{\z{\$}} &
	\bitbox{1}{\z{\textasciicircum D}}
\end{bytefield}
\end{center}

Draws an arbitrary bitmap image onto the matrix display starting at column
\Var*{pos}, encoded as per Table~\ref{tbl:int063}. A \Var*{pos} value of
``\z{\textasciitilde}'' represents the current column cursor position.

Each column data, from left to right, are given by \Var*{coldata} values,
each of which is a two-digit \ascii\ hexadecimal value with the 
least-significant bit representing the top row of the matrix.

The column data values are terminated by either a dollar-sign (``\z{\$}'') character or the
escape control character (hex byte \z{1B}), indicated in the protocol diagram above
simply as ``\z{\$}''.

The column cursor is moved to be after the end of the image.

If \Var*{merge} is the character ``\z.'' then each column's contents is cleared before
setting the pixels as per the \Var*{coldata} value. If \Var*{merge} is ``\z{M}'' the bits
set in \Var*{coldata} are added to the lit pixels already in the column.

The \Var*{trans} value indicates the transition effect to use when adding the image
to the display. See Table~\ref{tbl:transitions}.
\begin{table}
	\begin{center}
		\begin{tabular}{cl}\toprule
			\multicolumn{1}{c}{\bfseries Code} &
			\multicolumn{1}{c}{\bfseries Transition} \\\midrule
			\z. & No transition \\
			\z> & Scroll in from left \\
			\z< & Scroll in from right \\
			\z\textasciicircum & Scroll up from bottom \\
			\z{v} & Scroll down from top \\
			\z{L} & wipe left \\
			\z{R} & wipe right \\
			\z{U} & wipe up \\
			\z{D} & wipe down \\
			\z{|} & wipe left and right from middle column\\
			\z{-} & wipe up and down from middle row\\
			\z{?} & choose a random transition \\
			\bottomrule
		\end{tabular}
		\caption{Transition Effect Codes\label{tbl:transitions}}
	\end{center}
\end{table}

\section{\z{L}---Light Multiple \led s}
\begin{center}
\begin{bytefield}[endianness=little,bitwidth=0.11111\textwidth]{8}
%	\bitheader{0-7} \\
	\bitbox[]{1}{\scriptsize0}&
	\bitbox[]{1}{\scriptsize1}&
	\bitbox[]{1}{\scriptsize2}&
	\bitbox[]{1}{\scriptsize3}&
	\bitbox[]{1}{\dots}&
	\bitbox[]{1}{\scriptsize$n$}&
	\bitbox[]{1}{\scriptsize$n+1$}&
	\bitbox[]{1}{\scriptsize$n+2$}\\
	\bitbox{1}{\z{L}} &
	\bitbox[tbl]{1}{\Var*{led$_0$}} &
	\bitbox[tb]{1}{\Var*{led$_1$}} &
	\bitbox[tb]{1}{\Var*{led$_2$}} &
	\bitbox[tb]{1}{$\cdots$}&
	\bitbox[tb]{1}{\Var*{led$_{n-1}$}} &
	\bitbox{1}{\z{\$}} &
	\bitbox{1}{\z{\textasciicircum D}}
\end{bytefield}
\end{center}

This command is identical to the \z{S} command (see below), except that multiple
discrete \led s can be specified, all of which are illuminated simultaneously.
See Table~\ref{tbl:lightcodes}. Note that if a strobe sequence is running (via a previous \z{*} command), it remains running.

The list of \Var*{led} values is terminated by either a dollar sign (\z{\$}) character
or an escape byte (hex value \z{1B}), represented in the protocol diagram as ``\z{\$}''.

\section{\z{Q}---Query Readerboard Status}
\begin{center}
\begin{bytefield}[endianness=little,bitwidth=0.11111\textwidth]{2}
	\bitheader{0-1} \\
	\bitbox{1}{\z{Q}} &
	\bitbox{1}{\z{\textasciicircum D}}
\end{bytefield}
\end{center}

This command causes the sign to send a status report back to the host to indicate
the general status of the device except for the discrete \led\ display which
may be queried using the \z? command. The response has the form:

\medskip

\begin{center}
\begin{bytefield}[endianness=little,bitwidth=0.11111\textwidth]{9}
%	\bitheader{0-8} \\
	\bitbox[]{1}{\scriptsize0}&
	\bitbox[]{1}{\scriptsize1}&
	\bitbox[]{1}{\scriptsize2}&
	\bitbox[]{1}{\scriptsize3}&
	\bitbox[]{1}{\scriptsize4}&
	\bitbox[]{1}{\scriptsize5}&
	\bitbox[]{1}{\scriptsize6}&
	\bitbox[]{1}{\scriptsize7}&
	\bitbox[]{1}{\dots}\\
	\bitbox{1}{\z{Q}} &
	\bitbox{1}{\Var*{model}} &
	\bitbox{1}{\z{=}} &
	\bitbox{1}{\Var*{ad}}&
	\bitbox{1}{\Var*{uspd}}&
	\bitbox{1}{\Var*{rspd}}&
	\bitbox{1}{\z{V}} &
	\bitbox{2}{\Var*{hwversion}}\\
	\bitbox{1}{\z\$}&
	\bitbox{1}{\z{R}} &
	\bitbox{2}{\Var*{romversion}}&
	\bitbox{1}{\z{\$}}& 
	\bitbox{1}{\z{S}}& 
	\bitbox{3}{\Var*{serial}} \\
	\bitbox{1}{\z{\$}} &
	\bitbox{1}{\z{M}} &
	\bitbox[tbl]{2}{\Var*{coldata$_0$}} &
	\bitbox[tb]{2}{\Var*{coldata$_1$}} &
	\bitbox[tb]{1}{$\cdots$} &
	\bitbox[tbr]{2}{\Var*{coldata$_{63}$}}\\
	\bitbox{1}{\z{\textbackslash n}}
\end{bytefield}
\end{center}

The \Var*{model} field may be ``\z{L}'' for the legacy hardware the author still has
lying around (but this shouldn't be something anyone else would need to see), or
``\z{M}'' for the current 64$\times$8 matrix display hardware.

\Var*{hwversion} and \Var*{romversion} indicate the versions, respectively, of the 
hardware the firmware was compiled to drive, and of the firmware itself. Each of
these fields are variable-width and conform to the semantic version standard 2.0.0.\footnote{See
\href{https://semver.org}{semver.org}.} Each is terminated by a dollar-sign (\z\$) character (and
thus those strings may not contain dollar signs).

The \Var*{serial} field is a variable-width alphanumeric string which was set when the firm\-ware was
compiled. It should be a unique serial number for the device (although that depends on
some effort on the part of the person compiling the firmware to insert that serial number
each time). Serial numbers 0--299 are reserved for the original author's use. This string is also
terminated with a dollar sign.

The \Var*{coldata} bytes are sent just as with the \z{I} command, as hexadecimal values indicating
the \led s lit on the matrix display, with \Var*{coldata$_0$} being the leftmost column of the display
and \Var*{coldata$_{63}$} being the rightmost. In each column, the least significant bit indicates
the \led\ on the top row.

The \Var*{ad}, \Var*{uspd}, and \Var*{rspd} values are as last set by the \z= command (or the factory
defaults if they were never changed).

The status message sent to the host is terminated by a newline character (hex byte \z{0A}),
indicated in the protocol description above as ``\z{\textbackslash n}''.

\section{\z{S}---Light Single \led}
\begin{center}
\begin{bytefield}[endianness=little,bitwidth=0.11111\textwidth]{3}
	\bitheader{0-2} \\
	\bitbox{1}{\z{S}} &
	\bitbox{1}{\Var*{led}} &
	\bitbox{1}{\z{\textasciicircum D}}
\end{bytefield}
\end{center}

Stops the flasher (cancelling any previous \z{F} command) and turns off all discrete
\led s. The single \led\ indicated by \Var*{led} is turned on. See Table~\ref{tbl:lightcodes}. Note that if a strobe sequence is running (via a previous \z{*} command),
it remains running.

This command may be given in upper- or lower-case (``\z{s}'' or ``\z{S}'').

\section{\z{T}---Display Text}
\begin{center}
\begin{bytefield}[endianness=little,bitwidth=0.11111\textwidth]{9}
%	\bitheader{0-8} \\
	\bitbox[]{1}{\scriptsize0}&
	\bitbox[]{1}{\scriptsize1}&
	\bitbox[]{1}{\scriptsize2}&
	\bitbox[]{1}{\scriptsize3}&
	\bitbox[]{1}{\scriptsize4}&
	\bitbox[]{2}{\dots}&
	\bitbox[]{1}{\scriptsize$n$+4}&
	\bitbox[]{1}{\scriptsize$n$+5}\\
	\bitbox{1}{\z{T}} &
	\bitbox{1}{\Var*{merge}}&
	\bitbox{1}{\Var*{align}}&
	\bitbox{1}{\Var*{trans}}&
	\bitbox{3}{\Var*{string}}&
	\bitbox{1}{\z{ESC}}&
	\bitbox{1}{\z{\textasciicircum D}}
\end{bytefield}
\end{center}

Displays the text \Var*{string} at the current cursor position. The cursor position is
then moved past the text to be ready for the next string to be printed.

The \Var*{align} code indicates how the text is aligned on the display (TBD).

The \Var*{trans} code specifies the transition effect to be used to add this string to the
display as shown in Table~\ref{tbl:transitions}.

The text is rendered in the current font and may contain any 8-bit bytes except as otherwise
noted (but avoiding \ascii\ control codes is wise to be safe from conflict with
future control codes which may be added to the protocol). The string is terminated by an
escape character (hex byte \z{1B}), indicated in the protocol description as \z{ESC}.
(Since a dollar sign may appear in \Var*{string}, the terminator must be an escape
character for this command.)

The string may include control codes as listed in Table~\ref{tbl:controlcodes}.

\section{\z{X}---Turn off Discrete \led s}
\begin{center}
\begin{bytefield}[endianness=little,bitwidth=0.11111\textwidth]{2}
	\bitheader{0-1} \\
	\bitbox{1}{\z{X}} &
	\bitbox{1}{\z{\textasciicircum D}}
\end{bytefield}
\end{center}

Turns off the flasher, strober, and all discrete \led s.

This command may be given in upper- or lower-case (``\z{x}'' or ``\z{X}'').

\indexintoc

\printindex
\end{document}
